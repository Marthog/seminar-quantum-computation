\documentclass{acm_proc_article-sp}
\usepackage[ngerman]{babel}
\usepackage[utf8]{inputenc}
\usepackage[T1]{fontenc}
\usepackage{graphicx} 			%Grafiken
\usepackage{xcolor}   			%Farbige Schrift ermöglichen
\usepackage{amsmath}  			%Zusätzliche Mathebefehle
\usepackage{mathtools} 
\usepackage[figurename=Figur,labelfont=bf]{caption}

\usepackage{tikz} 
\usetikzlibrary{backgrounds,fit,decorations.pathreplacing, positioning, calc}

\newcommand{\qreg}[1]{\ensuremath{\left|#1\right\rangle}}
\newcommand{\qregfun}[2]{\qreg{#1} \rightarrow \qreg{#2}}
\newcommand{\qregv}[2]{#1_{#2} |#2\rangle}

%\renewcommand{\figurename}{Fig.}

% `operator' will only be used by Hadamard (H) gates here.
% `phase' is used für controlled phase gates (dots).
% `surround' is used für the background box.
\tikzstyle{operator} = [draw,fill=white,minimum size=1.5em] 
\tikzstyle{cnot} = [draw,circle,minimum size=0.5em]  % TODO: das Kreuz in der Mitte
\tikzstyle{phase} = [fill,shape=circle,minimum size=5pt,inner sep=0pt]
\tikzstyle{surround} = [draw=black]
\tikzset{%
  block/.style    = {draw, thick, rectangle, minimum height = 4em,
    minimum width = 4em},
    }
\DeclareMathOperator{\NOT}{NOT}
\DeclareMathOperator{\CNOT}{CNOT}

\title{Quantenberechnungen - Einführung}
\author{Gunnar Bergmann}
\date{16.6.2017}

\newcommand{\pictureDeutschAlgo}{
    \begin{tikzpicture}
    % Qubits
        \node[left] at (0,0) (q1) {\qreg 0};
        \node[left] at (0,-1) (q2) { \qreg 1 };
        %
        \node[operator, right=of q1] (h11) {$H$} edge [-] (q1);
        \node[operator, right=of q2] (h12) {$H$} edge [-] (q2);

        \node[right, right=of h11] (uf11)  {$x$}     edge [-] (h11);
        \node[right, right=of h12] (uf12)  {$y$}    edge [-] (h12);
        \node[right=5em of uf11, left] (uf21)  {$x$};
        \node[right=5em of uf12, left] (uf22)   {$y \oplus f(x)$};

        \node (text) at ($(uf22)!0.5!(uf11)$) {$U_f$};

        \node[operator, right=of uf12] (h21) at (5,0) {$H$} edge [-] (uf21);
        \node[right] (out1) at (8,0) {} edge [-] (h21);
        \node[right] (out2) at (8,-1) {} edge [-] (uf22);

        \begin{pgfonlayer}{background} 
        \node[surround] (background) [fit = (uf11) (uf12) (uf21)] {};
        \end{pgfonlayer}
        %
    \end{tikzpicture}
    }

\newcommand{\pictureCNOTa}{
      \begin{tikzpicture}
    %
    %
    % Qubits
        \node[left] at (0,0) (q1) {\qreg{\psi}};
        \node[left] at (0,-1) (q2) {\qreg{\varphi}};
        %
        % Column 3
        \node[phase] (phase11) at (1,0) {} edge [-] (q1);
        \node[operator] (phase12) at (1,-1) {$X$} edge [-] (q2);
        \draw[-] (phase11) -- (phase12);
        %
        % Column 4
        \node[right] (out1) at (2,0) {\qreg{\psi}} edge [-] (phase11);
        \node[right] (out2) at (2,-1) {$\qreg{\varphi} \oplus \qreg{\psi}$} edge [-] (phase12);
        %
    \end{tikzpicture}
    }


\newcommand{\pictureCNOTb}{
    \begin{tikzpicture}[scale=1.0]
    %
    %
    % Qubits
        \node[left] at (0,0) (q1) {\qreg{\psi}};
        \node[left] at (0,-1) (q2) {\qreg{\varphi}};
        %
        % Column 3
        \node[phase] (phase11) at (1,0) {} edge [-] (q1);
        \node[cnot] (phase12) at (1,-1) {} edge [-] (q2);
        \draw[-] (phase11) -- (phase12);
        %
        % Column 4
        \node[right] (out1) at (2,0) {\qreg{\psi}} edge [-] (phase11);
        \node[right] (out2) at (2,-1) {$\qreg{\varphi} \oplus \qreg{\psi}$} edge [-] (phase12);
        %
    \end{tikzpicture}
    }

\newcommand{\pictureHadamard}{
      \begin{tikzpicture}[scale=1.0]
        \node at (0,0) (q1) {};
        \node[operator] at (1,0) (h1) {$H$};
        \node at (2,0) (out1) {};

          \draw[-] (q1) -- (h1) -- (out1);
        %
    \end{tikzpicture}
    }

\newcommand{\pictureToffoli}{
    \begin{tikzpicture}[scale=1.0]
    %
    %
    % Qubits
        \node[left] at (0,0) (q1) {$a$};
        \node[left] at (0,-1) (q2) {$b$};
        \node[left] at (0,-2) (q3) {$c$};
        %
        % Column 3
        \node[phase] (phase11) at (1,0) {} edge [-] (q1);
        \node[phase] (phase12) at (1,-1) {} edge [-] (q2);
        \node[cnot] (phase13) at (1,-2) {} edge [-] (q3);
        \draw[-] (phase11) -- (phase12) -- (phase13);
        %
        % Column 4
        \node[right] (out1) at (2,0) {$a$} edge [-] (phase11);
        \node[right] (out2) at (2,-1) {$b$} edge [-] (phase12);
        \node[right] (out3) at (2,-2) {$c \oplus ab$} edge [-] (phase13);
        %
    \end{tikzpicture}
    }


\begin{document}


\title{Quantenberechnungen - Einführung}
\subtitle{[Seminar Computing Beyond Turing]}
\numberofauthors{1} 
\author{\alignauthor
Gunnar Bergmann\\
       \affaddr{Universität zu Lübeck}\\ 
       \affaddr{Ratzeburger Allee 160}\\
       \affaddr{23562 Lübeck, Deutschland}\\ 
       \email{gunnar.bergmann@student.uni-luebeck.de}
}

\date{16.6.2017}

\maketitle
\begin{abstract}

In dieser Ausarbeitung werden die Grundlagen über Quantencomputer vorgestellt.
Es werden grundlegende Begriffe und mathematische Formulierungen zur Beschreibung von
Quantenrechner vorgestellt.
Es wird eine Einteilung der Berechenbarkeit und Komplexität gegeben und am Beispiel von Deutschs Algorithmuses erläutert, wie man die Eigenschaften von Quanten ausnutzen kann, um Berechnungen zu parallelisieren.
Desweiteren werden einige Probleme erklärt, die die Konstruktion von Quantenalgorithmen erschweren.

\end{abstract}

\section{Einführung}


In der Quantenmechanik sind Quanten nicht auf klassische Zustände beschränkt, sondern können auch Superpositionen einnehmen.
Das bedeutet, sie sind in mehreren Zuständen gleichzeitig. Quanten in Superposition können immer noch mit anderen
Quantensystemen interagieren, aber es ist nicht möglich, den Zustand direkt zu messen. Stattdessen wechselt das 
Quantensystem beim Messen in einen klassischen Zustand.

Quantencomputer benutzen die Eigenschaften der Quantenmechanik, um über Quantenparallelismus bei bestimmten Problemen
schneller rechnen zu können, als klassische Computer.

Wir werden sehen, dass Quantenrechner jeden anderen Rechner simulieren können, aber bestimmte Algorithmen Quantenparallelismus ausnutzen können, um exponentiell viele gleichzeitige Berechnungen durchzuführen.

Quantenmechanik hat oft unintuitive Effekte, aber mathematische Modelle erlauben
eine akkurate Beschreibung und Vorhersage von beobachtbarem Verhalten.
Diese unintuitive Natur von Quanten erschwert die Konstruktion von Quantenalgorithmen.

Es gibt nur wenige nützliche Algorithmen. Am wichtigsten ist davon die Simulation von Quantensystemen zu
Forschungszwecken, Shors Algorithmus zur Faktorisierung in Polynomialzeit und Grovers Algorithmus, um
einen Suchraum in Laufzeit $\mathcal{O}(\sqrt N)$ zu durchsuchen.

Alle diese Algorithmen gehen über dieses Paper hinaus. Stattdessen wird hier ein künstliches Problem, ohne
reale Anwendungsgebiete, betrachtet, dass in Exponentialzeit auf einem klassischen Rechner, aber
in Linearzeit auf einem Quantenrechner lösen lässt

\section{Quantenbits}

Quantenbits (gewöhnlich Qubits genannt) generalizieren klassische Bits.
Es gibt zwei Basiszustände, $\qreg 0$ und $\qreg 1$, die man wie klassische Bits interpretieren und verwenden kann.

Anders als gewöhnliche Bits erlauben Qubits Superpositionen, die Anteile von beiden Zuständen gleichzeitige enthalten.

Ein mathematisches Modell für ein Qubit ist ein Vektorraum
    \[\qreg \psi = a \qreg 0 + b \qreg 1 =
    \begin{pmatrix}
        a \\ b
    \end{pmatrix}
    \]
mit  \(a, b \in \mathbb C \) und $|a|^2 + |b|^2 = 1$, wobei $\qreg 0$ und $\qreg 1$ eine Orthonormalbasis bilden.

Durch geeignete Wahl von $a$ und $b$ können dann die Superpositionen modelliert werden.
Häufig tritt zum Beispiel der Zustand $\qreg + = \frac{\qreg 0 + \qreg 1}{\sqrt 2}$ auf.

Es ist nicht möglich, den exakten Zustand eines Qubits zu bestimmen.
Stattdessen ergibt eine Messung $\qreg 0$ mit Wahrscheinlichkeit $a^2$ und
$\qreg 1$ mit Wahrscheinlichkeit $b^2$.
Darüber hinaus verändert die Messung den Zustand des Qubits selber und
das Qubit nimmt dann den Zustand $\qreg 0$ bzw. $\qreg 1$ an.

Durch Messung kann man also nicht den genauen Zustand bestimmen und erhält anschließend nur noch die Basiszustände,
die keine weiteren Infürmationen als klassische Bits enthalten.
Wie in \ref{DeutschsAlgo} noch an einem Beispiel gezeigt wird, kann man aber mit Qubits in Superpositionen
weiterrechnen und so Quantenparallelismus erhalten, indem mit beiden Zuständen gleichzeitig gerechnet wird.

Theoretisch kann man Qubits auch in andere Basen als $\qreg 0$ und $\qreg 1$ angeben und messen. Sämtliche Eigenschaften
bleiben erhalten, aber der Einfachheit halber beschränkt man sich üblicherweise auf $\qreg 0$ und $\qreg 1$.


Betrachtet man ein System aus mehreren Qubits, bezeichnet man dieses als Quantenregister und stellt es
gut lesbar in einer Formel dar.

Zum Beispiel besteht das Quantenregister 
\[
    \qreg \psi = \qregv a {00} + \qregv a {01} + \qregv a {10} + \qregv a {11} = 
\begin{pmatrix}
    a_{00}  \\
    a_{01}  \\
    a_{10}  \\
    a_{11}  \\
\end{pmatrix}
\]
mit
\[
    |a_{00}|^2 + |a_{01}|^2 + |a_{10}|^2 + |a_{11}|^2 = 1
\]
aus zwei Qubits.

Verallgemeinert ist also das Quantenregister aus $n$ Qubits durch
\[
    \qreg \psi = \sum_{x \in {\{0,1\}}^n} \qregv c {x},
    \sum_{x \in {\{0,1\}}^n} |c_x|^2 = 1
\]
definiert und hat eine Gesamtanzahl von $2^n$ mögliche Zustände.

\section{Quantenschaltkreise}
\label{qcircuits}

Einige Quantengatter können über das Generalisieren von klassischen Gattern erzeugt werden.

Eines der einfachsten Schaltkreise ist das NOT-Gatter, das als Qubit folgendermaßen aussieht:
    \[NOT(a \qreg 0 + b \qreg 1)  = b \qreg 0 + a \qreg 1\]

Es ist eine Verallgemeinerung des klassischen NOT-Gatters, denn offensichtlich gelten
\begin{eqnarray*}
    NOT(\qreg 0) = \qreg 1 \\
    NOT(\qreg 1) = \qreg 0
\end{eqnarray*}

und der Tausch der Komponenten verändert die Länge des Vektors nicht, sodass auch die Gesamtwahrscheinlichkeit
von $1 = |a|^2 + |b|^2$ erhalten bleibt.

Aufgrund von Eigenschaften, deren Umfang über diese Ausarbeitung hinus gehen, interagieren Quantensysteme immer 
in linearer Weise.

Im Falle von klassischen Gattern reicht eine Wahrheitstabelle, aber für Quantenrechner ist
dieses aufgrund der kontinuierlichen Zustandskombinationen nicht mehr möglich. Stattdessen werden Gatter über eine Matrix definiert,
sodass die Anwendung als Multiplikation mit dem Vektor des Quantenregisters beschrieben wird.

Diese Matrizen müssen die Länge des Vektors erhalten, damit sich die Wahrscheinlichkeiten auch nach der Anwendung 
noch auf 1 aufsummieren.

Dafür müssen die Matrizen unitär sein. Eine Matrix $U$ ist unitär, wenn für die konjugiert transponierte $U^{\dagger}$ das
Inverse ist, also $U^\dagger \cdot U = U \cdot U^\dagger = I$, wobei I die Einheitsmatrix entsprechender Größe bezeichnet.
Dieses bedeutet auch, dass es zu jedem Quantengatter ein Inverses gibt und somit jede Berechnung umkehrbar ist.

Wenn man die Anwendung des NOT-Gatter als Matrix-Vektor-Multiplikation schreibt, erhält man:
\[
\begin{bmatrix}
    0 & 1 \\
    1 & 0 \\
\end{bmatrix} \cdot 
\begin{pmatrix}
    a \\ b
\end{pmatrix} = 
\begin{pmatrix}
    b \\ a
\end{pmatrix} 
\]

Offensichtlich ist $\NOT^\dagger = \NOT$ das Inverse.

Da jedes Gatter invertiert werden kann, muss Eingabe- und Ausgabegröße gleich sein. 
Aufgrund dieser Einschränkung ist es nicht möglich, Bits zu kopieren, wie es im Fall klassischer Bits durch 
Verbinden von Kabeln sehr einfach möglich ist. 
Diese Operation ist auf Quantenrechnern im Allgemeinen nicht mehr möglich und wird in 
\ref{noclone} noch genauer untersucht.

Deswegen sind auch bestimmte grundlegende Operationen wie AND, OR und XOR nicht ohne weiteres möglich. 
Dieses lässt sich aber über einfügen weiterer Eingaben, sogenannter Kontrollbits, umgehen.

Ein Beispiel ist das CNOT. Es ist eine Verallgemeinerung des XOR-Gatters, stellt die Funktion
\[
    \qregfun{a, b}{a \oplus b, b}
\]

dar und hat die Matrix
\[
\begin{bmatrix}
    1 & 0 & 0 & 0 \\
    0 & 1 & 0 & 0 \\
    0 & 0 & 0 & 1 \\
    0 & 0 & 1 & 0 \\
\end{bmatrix} 
\]

CNOT ist unitär, da $\CNOT \cdot \CNOT = I$.

\begin{figure}
  \centerline{
    \pictureCNOTa
    \hbox{=}
    \pictureCNOTb
  }
  \caption{
    Verschiedene Darstellungen des CNOT-Gatters. Control-Bits werden als Punkte gezeichnet.
  }
\end{figure}

Ähnliche Generalisierungen sind auch für andere Gatter möglich. In \ref{Simulate} wird gezeigt, wie jedes
klassische Gatter über eine bestimmte Schaltung simuliert werden kann.

% TODO: no loops

Ein weiteres, wichtiges Gatter für viele Algorithmen ist das Hadamard-Gatter
\[
    H = \frac{1}{\sqrt 2}
    \begin{bmatrix}
        1 & 1 \\
        1 & -1 \\
    \end{bmatrix}
\]

\begin{center}
    \pictureHadamard
    %\captionof{figure}{Darstellung des Hadamard-Gatters im Schaltplan.}
\end{center}

Auf den Eingaben \qreg 0 und \qreg 1 mischt das Hadamard-Gatter die beiden
Zustände zu gleichen Anteilen.

\[
    \qreg + = H \cdot \qreg 1 = H \cdot 
    \begin{pmatrix}
        1 \\ 0
    \end{pmatrix}
    = 
    \frac{1}{\sqrt 2}
    \begin{pmatrix}
        1 \\ 1
    \end{pmatrix}
    = 
    \frac{\qreg 0 + \qreg 1}{\sqrt 2}
\]

\[
    \qreg - = H \cdot \qreg 0 = H \cdot 
    \begin{pmatrix}
        0 \\ 1
    \end{pmatrix}
    = 
    \frac{1}{\sqrt 2}
    \begin{pmatrix}
        1 \\ -1
    \end{pmatrix}
    = 
    \frac{\qreg 0 - \qreg 1}{\sqrt 2}
\]

Beim Messen von \qreg + und \qreg -  erhält man 0 und 1 mit jeweils gleichen Wahrscheinlichkeiten.
Hierbei kann man erkennen, dass es auch durch wiederholte
Messungen nicht ohne weiteres möglich ist, die beiden Zustände zu unterscheiden.

\section{No cloning theorem}
\label{noclone}

In klassischen Schaltkreisen kann ein Bit einfach kopiert werden indem man mehrere Kabel befestigt. Dieses ist
bei Quantenbits nicht mehr möglich. Es gibt keinen Weg, um im Allgemeinen ein Bit zu klonen.

Jedes Quantengatter ist reversibel, da die zugehörige Matrix unitär sein muss. 
Eine Klon-funktion wie $\qregfun{\varphi\psi}{\psi\psi}$ ist nicht reversibel und damit ist Klonen von Quanten über
Quantengatter nicht möglich. Auch sonst ist nicht möglich Quanten zu klonen.

Dieses kann wie ein Widerspruch mit dem vorher definierten CNOT aussehen, denn für $a=\qreg 0$ erhält man
genau $\qregfun{0, \psi}{\psi, \psi}$.
Es ist allerdings kein Widerspruch, wie recht einfach gezeigt werden kann:
Für $\psi = a \qreg 0 + b \qreg 1$ ergibt die Anwendung von vorherigen CNOT-Gattern genau die Ausgabe
\[
    a \qreg{00} + b \qreg{11} 
\]
aber für zwei komplett unabhängige Quanten mit Wert $\psi$ erhält man
\[
    \qreg{\psi \psi} = a^2 \qreg{00} + ab \qreg{01} + ab \qreg{10} + b^2 \qreg{11} 
\]

Dieses ist nur ein Klon für $a=0$ und $b=0$.
Spezifische Bits können also sehr wohl geklont werden, allerdings übersteigt der dabei kopierte Informationsgehalt
nicht denen klassischer Bits.

\section{Simulation klassischer Computer}
\label{Simulate}

Das Toffoli-Gatter spielt eine zentrale Role in der Simulation klassischer Computer.
Es berechnet die Funktion.
\[
    \qregfun{a, b, c}{a, b, c\oplus (a\wedge b)}
\]

\begin{center}
    \pictureToffoli
    \captionof{figure}{Toffoli-Gatter}
\end{center}


Das Toffoli-Gatter hat ein Inverses. Genauer gesegt, ist es sein eigenes Inverses, denn zweimal Anwenden
ergibt
\[
    \qregfun{a, b, c\oplus (a\wedge b) \oplus (a\wedge b)}{a, b, c}
\]


Eine Menge von Toffoli-Gattern kann jeden klassischen Schaltkreis simulieren. Für diese werden nur die QBits
$\qreg 0$ und $\qreg 1$ verwendet, aber keine weiteren Quantenzustände.

Als erstes zeigen wir, dass das Toffoli-gatter zum Kopieren dieser Bits eingesetzt werden kann.
In \ref{noclone} haben wir gezeigt, dass das CNOT-Gatter trotz des \emph{no-cloning-theorems} klassische
Bits klonen kann. 
Setzt man $a = \qreg 1, c = \qreg 0$, erhält man die Funktion $\qregfun{1,b,0}{1,b,b}$,

Wenn man stattdessen $c = \qreg 1$ setzt, erhält man $\qregfun{a,b,1}{a,b, \neg (a\wedge b)}$
und kann über dieses das NAND-Gatter simulieren
NAND ist ein universelles Gatter, d.h. man kann jedes andere logische Gatter über Folgen von NANDs realisiert werden.

Damit haben wir gezeigt, dass jeder klassischer Schaltkreis und damit auch klassischer Computer nur mit Toffoli-Gatter
ohne Laufzeitverlust simuliert werden kann.

\section{Deutschs Algorithmus}
\label{DeutschsAlgo}

Deutschs Algorithmus ist ein einfacher Algorithmus, der Quantenparallelismus zeigen kann. 
Hier wird eine vereinfachte Variante gezeigt.

Sei dazu $ f(x) : \{0,1\} \rightarrow \{0,1\} $ eine beliebige Funktion. $f$ ist unbekannt, es kann aber
ein Quantenschaltkreis dazu konstruiert werden.

Deutschs Algorithmus testet nun, ob $f(0) \neq f(1)$, bzw. $f(1) \oplus f(0)$. Dieses könnte ein klassischer
Rechner durch Auswertung der beiden Funktionsaufrufe sehr einfach entscheiden, aber hier werden Eigenschaften
der Quantenmechanik benutzt, um $f$ nur einmal auszuwerten.

Dazu wird als erstes ein Quantenschaltkreis $U_f$ auf Zweibitregistern konstruiert, der die Funktion 
$\qregfun{x,y}{x, y \oplus f(x)}$ umsetzt. 

Durch $y = \qreg 0$ kann offensichtlich $f(x)$ berechnet werden. Durch ein Hadamard-Gatter und kann man nun 
$x = \qreg + = \frac{\qreg 0 + \qreg 1}{\sqrt 2}$ setzen
und wendet dann $U_f$ an. Das Ergebnis ist
\[
    \qreg{x, f(x)} = \frac{\qreg{0, f(0)} + \qreg{1, f(0)}}{\sqrt 2}
\]

Hier kann man bereits sehen, dass ein einzelner Schaltkreis ausreicht, um über Quantenparallelismus sowohl $f(0)$, als auch $f(1)$ 
zu berechnen, allerdings kann man dieses Ergebnis noch nicht nutzen, da man beim Messen der Ausgabe jeweils nur \qreg{0, f(0)} oder
\qreg{1, f(1)} erhält.

\newcommand{\qbitH}[1]{\left[\frac{\qreg 0 #1 \qreg 1}{\sqrt 2}\right]}

Bei Deutschs Algorithmus wird nun eine weitere Eigenschaft der Quantenmechanik, die
sogenannte Interferenz genutzt. Dafür wird über ein weiteres Hadamard-Gatter $y$ auf $\qreg - = \frac{\qreg 0 - \qreg 1}{\sqrt 2}$
gesetzt. Anwenden von $U_f$ ergibt
\begin{equation}
    U_f \cdot \left( \qreg x \qbitH - \right) = (-1)^{f(x)} \qreg x \qbitH - 
\end{equation}

Nach diesen Vorüberlegungen kann der eigentliche Algorithmus beginnen.
Als erstes wird ein Quantenregister \qreg{01} über zwei Hadamard-Gatter zu 
\begin{equation}
\qreg {\psi_1} = \qbitH+\qbitH- 
\end{equation}
transformiert.

Anwendung von $U_f$ ergibt 
\begin{equation}
    \qreg{\psi_2} = \begin{cases}
        \pm \qbitH+ \qbitH- & \text{für } f(0) = f(1) \\[1.5em]
        \pm \qbitH- \qbitH- & \text{für } f(0) \neq f(1)
        \end{cases}
\end{equation}

Anwendung eines weiteren Hadamard-Gatters ergibt
\begin{equation}
    \begin{split}
        \qreg{\psi_3} &= \left\{\begin{array}{lr}
            \pm \qreg 0 \qbitH- & \text{für } f(0) = f(1) \\[1.5em]
            \pm \qreg 1 \qbitH- & \text{für } f(0) \neq f(1) 
        \end{array}\right\} \\[1.5em] &= \pm \qreg{f(0) \oplus f(1)} \qbitH+
    \end{split}
\end{equation}

Durch Messung des ersten Qubits erhält man nun $f(0) \oplus f(1)$, wobei $f$ während des
ganzen Durchganges nur einmal ausgewertet wurde.

\begin{figure}
    \pictureDeutschAlgo
    \caption{figure}{Deutschs Algorithmus. $H$ symbolisiert Hadamard-Gatter.}
\end{figure}

\section{Deutsch-Jozsa Algorithmus}

Der Deutsch-Jozsa Algorithmus ist eine Verallgemeinerung von Deutschs Algorithmus.
Gegeben eine n-bit Funktion $f : \{0,1\}^n \rightarrow \{0,1\} $, der Algorithmus entscheidet, ob $f$ eine konstante Funktion
ist (immer den gleichen Wert zurückgibt) oder balanziert (in der Hälfte der Fälle 0, in der anderen 1).

Die Lösung des Problems ist analog zu Deutschs Algorithmus, aber verwendet statt $x$ $n$ Qubits, die alle über Hadamard-Gatter
in Superposition gebracht werden.

Obwohl das Problem an sich noch keinen realen Nutzen hat, existiert ein Quantenalgorithmus, der die Antwort in einer einfachen
Auswertung von $f$ bestimmt, während eine klassische Maschine $\Omega(2^n)$ Auswertungen braucht.

\section{Komplexität}

Wie in \ref{qcircuits} gezeigt wurde, kann jeder klassische Computer ohne Effizienzverlust simuliert werden.

Umgedreht kann auch ein Quantenalgorithmus auf einem klassischen Computer simuliert werden, indem
die Komponenten der Quantenregister einzeln berechnet werden. 

Von der Berechenbarkeit gibt es also keinen Unterschied, aber wenn man die effizient berechenbaren Probleme betrachtet,
ist der Quantenrechner über Quantenparallelismus möglicherweise effizienter.

Die Klasse $BQP$ (bounded error quantum polynomial time) bezeichnet die Klasse der Probleme, die ein Quantenrechner in 
Polynomialzeit berechnen kann.

Wie in \ref{qcircuits} gezeigt, gilt $P \subseteq BQP$.

Da Quantenrechner über gute Zufallsquellen verfügen, gilt zudem $BPP \subseteq BQP$.

Zusätzlich kann kein Problem, dass nicht in $PSPACE$ liegt effizient in Polynomialzeit gelöst werden.
Damit gilt also $BQP \subseteq PSPACE$. 

Sollte $P \neq BQP$ gezeigt werden, dann folgt auch $ P \neq PSPACE$.
Letzteres ist aber ein noch offenes Problem, zu dem schon viele Lösungsversuche unternommen wurden.
Deswegen ist anzunehmen, dass die genaue Beziehung zwischen $BQP$ und $PSPACE$ auch noch länger offen bleibt.
Ähnliches gilt für die Untergrenze, denn sollte $BQP = P$ gelten, dann gäbe es mit Shors Algorithmus auch einen 
effizienten Algorithmus für die Primfaktorzerlegung.


% TODO: warum

Die Beziehung zwischen NP und $BQP$ ist nicht bekannt. Einige Probleme wie Primfaktorzerlegung
können zwar in Polynomialzeit von einem Quantenrechner gelöst werden,
aber bisher wurde noch für kein NP-hartes Problem ein Algorithmus angegeben.

Da Quantenberechnungen häufig randomisierte Algorithmen verwenden, wäre es angebracht, $BQP$ mit $BPP$ zu vergleichen,
aber dieses wurde bisher nicht. 

\section{Zusammenfassung und Ausblick}

Wir haben gesehen, dass es möglich ist, bestimmte Probleme über Quantenalgorithmen efficienter zu Lösen.

Die genaue Beziehung zwischen $BQP$ und $P$ bzw. $NP$ ist noch unklar.

Es sind bisher nur wenige Quantenalgorithmen verfügbar und die unintuitive Natur der Quantenmechanik
erschwert es weitere zu finden. 

Bisher wurden keine Quantencomputer in nutzbarer Größe gebaut. Wenn dies geschieht können Eigenschaften
mit bestehenden Computern weiter untersucht werden.

\bibliographystyle{abbrv}
\bibliography{sigproc}
\nocite{Barenco, Nielsen}

\end{document}
