\documentclass[xcolor=colortbl
%,draft
,ngerman
]{beamer}
\mode<presentation>

\usepackage[ngerman]{babel}
\usepackage[utf8]{inputenc}
\usepackage[T1]{fontenc}
\usepackage{graphicx} 			%Grafiken
\usepackage{xcolor}   			%Farbige Schrift ermöglichen
\usepackage{amsmath}  			%Zusätzliche Mathebefehle
\usepackage{mathtools} 
\usepackage{tikz}

\usetikzlibrary{backgrounds,fit,decorations.pathreplacing, positioning, calc}

\usepackage{tikz} 
\usetikzlibrary{backgrounds,fit,decorations.pathreplacing, positioning, calc}

\newcommand{\qreg}[1]{\ensuremath{\left|#1\right\rangle}}
\newcommand{\qregfun}[2]{\qreg{#1} \rightarrow \qreg{#2}}
\newcommand{\qregv}[2]{#1_{#2} |#2\rangle}
\newcommand{\qregPlus}{\frac{\qreg 0 + \qreg 1}{\sqrt 2}}
\newcommand{\qregMinus}{\frac{\qreg 0 - \qreg 1}{\sqrt 2}}
\newcommand{\qbitH}[1]{\left[\frac{\qreg 0 #1 \qreg 1}{\sqrt 2}\right]}


%\renewcommand{\figurename}{Fig.}

% `operator' will only be used by Hadamard (H) gates here.
% `phase' is used für controlled phase gates (dots).
% `surround' is used für the background box.
\tikzstyle{operator} = [draw,fill=white,minimum size=1.5em] 
\tikzstyle{cnot} = [minimum size=0.4em,label=center:$\oplus$]  % TODO: das Kreuz in der Mitte
\tikzstyle{phase} = [fill,shape=circle,minimum size=5pt,inner sep=0pt]
\tikzstyle{surround} = [draw=black]
\tikzset{%
  block/.style    = {draw, thick, rectangle, minimum height = 4em,
    minimum width = 4em},
    }
\DeclareMathOperator{\NOT}{NOT}
\DeclareMathOperator{\CNOT}{CNOT}

\DeclareMathOperator{\cP}{P}
\DeclareMathOperator{\cNP}{NP}
\DeclareMathOperator{\cPSPACE}{PSPACE}
\DeclareMathOperator{\cL}{L}
\DeclareMathOperator{\cBQP}{BQP}
\DeclareMathOperator{\cBPP}{BPP}

\title{Quantenberechnungen - Einführung}
\author{Gunnar Bergmann}
\date{16.6.2017}

\newcommand{\pictureDeutschAlgo}{
    \begin{tikzpicture}
    % Qubits
        \node[left] at (0,0) (q1) {\qreg 0};
        \node[left] at (0,-1) (q2) { \qreg 1 };
        %
        \node[operator, right=of q1] (h11) {$H$} edge [-] (q1);
        \node[operator, right=of q2] (h12) {$H$} edge [-] (q2);

        \node[right, right=of h11] (uf11)  {$x$}     edge [-] (h11);
        \node[right, right=of h12] (uf12)  {$y$}    edge [-] (h12);
        \node[right=5em of uf11, left] (uf21)  {$x$};
        \node[right=5em of uf12, left] (uf22)   {$y \oplus f(x)$};

        \node (text) at ($(uf22)!0.5!(uf11)$) {$U_f$};

        \node[operator, right=of uf12] (h21) at (5,0) {$H$} edge [-] (uf21);
        \node[right] (out1) at (8,0) {} edge [-] (h21);
        \node[right] (out2) at (8,-1) {} edge [-] (uf22);

        \begin{pgfonlayer}{background} 
        \node[surround] (background) [fit = (uf11) (uf12) (uf21)] {};
        \end{pgfonlayer}
        %
    \end{tikzpicture}
    }

    \newcommand{\CNOTFormula}{\qregfun{\psi, \varphi}{\psi, \varphi \oplus \psi}}
    \newcommand{\CNOTMatrix}{
                \begin{bmatrix}
                    1 & 0 & 0 & 0 \\
                    0 & 1 & 0 & 0 \\
                    0 & 0 & 0 & 1 \\
                    0 & 0 & 1 & 0 \\
                \end{bmatrix} 
            }

\newcommand{\pictureCNOTa}{
      \begin{tikzpicture}
    %
    %
    % Qubits
        \node[left] at (0,0) (q1) {\qreg{\psi}};
        \node[left] at (0,-1) (q2) {\qreg{\varphi}};
        %
        % Column 3
        \node[phase] (phase11) at (1,0) {} edge [-] (q1);
        \node[operator] (phase12) at (1,-1) {$X$} edge [-] (q2);
        \draw[-] (phase11) -- (phase12);
        %
        % Column 4
        \node[right] (out1) at (2,0) {\qreg{\psi}} edge [-] (phase11);
        \node[right] (out2) at (2,-1) {$\qreg{\varphi} \oplus \qreg{\psi}$} edge [-] (phase12);
        %
    \end{tikzpicture}
    }


\newcommand{\pictureCNOTb}{
    \begin{tikzpicture}[scale=1.0]
    %
    %
    % Qubits
        \node[left] at (0,0) (q1) {\qreg{\psi}};
        \node[left] at (0,-1) (q2) {\qreg{\varphi}};
        %
        % Column 3
        \node[phase] (phase11) at (1,0) {} edge [-] (q1);
        \node[cnot] (phase12) at (1,-1) {} edge [-] (q2);
        \draw[-] (phase11) -- (phase12);
        %
        % Column 4
        \node[right] (out1) at (2,0) {\qreg{\psi}} edge [-] (phase11);
        \node[right] (out2) at (2,-1) {$\qreg{\varphi} \oplus \qreg{\psi}$} edge [-] (phase12);
        %
    \end{tikzpicture}
    }

\newcommand{\pictureHadamard}{
      \begin{tikzpicture}[scale=1.0]
        \node at (0,0) (q1) {};
        \node[operator] at (1,0) (h1) {$H$};
        \node at (2,0) (out1) {};

          \draw[-] (q1) -- (h1) -- (out1);
        %
    \end{tikzpicture}
    }

\newcommand{\pictureToffoli}{
    \begin{tikzpicture}[scale=1.0]
    %
    %
    % Qubits
        \node[left] at (0,0) (q1) {$a$};
        \node[left] at (0,-1) (q2) {$b$};
        \node[left] at (0,-2) (q3) {$c$};
        %
        % Column 3
        \node[phase] (phase11) at (1,0) {} edge [-] (q1);
        \node[phase] (phase12) at (1,-1) {} edge [-] (q2);
        \node[cnot] (phase13) at (1,-2) {} edge [-] (q3);
        \draw[-] (phase11) -- (phase12) -- (phase13);
        %
        % Column 4
        \node[right] (out1) at (2,0) {$a$} edge [-] (phase11);
        \node[right] (out2) at (2,-1) {$b$} edge [-] (phase12);
        \node[right] (out3) at (2,-2) {$c \oplus ab$} edge [-] (phase13);
        %
    \end{tikzpicture}
    }

\newcommand{\hadamardPlus}{
    \ensuremath{
        \qreg + = H \cdot \qreg 0 = H \cdot 
        \begin{pmatrix}
            1 \\ 0
        \end{pmatrix}
        = 
        \frac{1}{\sqrt 2}
        \begin{pmatrix}
            1 \\ 1
        \end{pmatrix}
        = 
        \frac{\qreg 0 + \qreg 1}{\sqrt 2}
    }
}

\newcommand{\hadamardMinus}{
    \ensuremath{
        \qreg - = H \cdot \qreg 1 = H \cdot 
        \begin{pmatrix}
            0 \\ 1
        \end{pmatrix}
        = 
        \frac{1}{\sqrt 2}
        \begin{pmatrix}
            1 \\ -1
        \end{pmatrix}
        = 
        \frac{\qreg 0 - \qreg 1}{\sqrt 2}
    }
}



% Load Theme
\usetheme[slogan=false, navigation=false, 
frametotal=false,
myriad=false]{UzL}

\begin{document}
\maketitle

\iffalse
\begin{frame}
    \tableofcontents
\end{frame}
\fi

\section{Einführung}


\begin{frame}{Quantenberechnungen - Einführung}
    \begin{itemize}
        \item Grundlagen und mathematisches Modell
        \item Einschränkungen
        \item Simulation klassischer Computer
        \item Deutschs Algorithmus
        \item Komplexität
    \end{itemize}
\end{frame}

\begin{frame}{Quantencomputer}
    \begin{itemize}
        \item nutzen quantenmechanische Eigenschaften
        \item können klassische Rechner effizient simulieren
        \item für bestimmte Aufgaben effizienter
        \item Quantensysteme haben unintuitives Verhalten
        \item bisher nur wenige Algorithmen
    \end{itemize}
\end{frame}

\begin{frame}{Quantenalgorithmen}
    \begin{itemize}
        \item Simulation von Quantensystemen
        \item Shors Algorithmus für Primzahlzerlegung in Polynomialzeit
        \item Grovers Algorithmus für Suche in $\Theta(\sqrt N)$
        \item Deutschs Algorithmus
            \begin{itemize}
                \item einfaches Beispiel
                \item demonstriert Vorteile von Quantenalgorithmen
                \item später mehr dazu
            \end{itemize}
        \item Simons Problem zeigt Vorteile bei randomisierten Algorithmen
    \end{itemize}
\end{frame}

\begin{frame}{Quantenbits (Qubits)}
    \begin{itemize}
        \item Generalisierung von klassischen Bits
        \item Basiszustände \qreg 0 und \qreg 1
        \item Superposition: 
            \begin{itemize}
                \item Anteile von beiden Zuständen gleichzeitig
                \item kann nicht genau bestimmt werden
            \end{itemize}
    \end{itemize}
\end{frame}

\subsection{Quantenbits}
\begin{frame}{Quantenbits (Qubits)}
    \begin{block}{Notation}{Qubits}
        \begin{align*}
        &\qreg \psi = \alpha \qreg 0 + \beta \qreg 1 = \begin{pmatrix} \alpha \\ \beta \end{pmatrix}  \\
        &\alpha, \beta \in \mathbb C \\
        &|\alpha|^2 + |\beta|^2 = 1
        \end{align*}
    \end{block}
    Für einfache Fälle reicht oft auch $\alpha,\beta \in \mathbb R $.
\end{frame}

\begin{frame}{Quantenbits (Qubits)}
    \begin{itemize}
        \item exakter Zustand nicht ermittelbar
        \item Messung ergibt \qreg 0 mit Wahrscheinlichkeit $|\alpha|^2$ und \qreg 1 mit $|\beta|^2$
        \item Messung verändert den Zustand zu \qreg 0 oder \qreg 1
    \end{itemize}
\end{frame}

\begin{frame}{Quantenregister}
    \begin{itemize}
        \item mehrere Qubits
        \item stellt Gesamtzustand aller Qubits dar
        \item Operationen werden auf ganzen Registern statt einzelnen Bits definiert.
    \end{itemize}
\end{frame}

\begin{frame}{Quantenregister}
    \begin{example}
    \[
    \qreg \psi = \qregv \alpha {00} + \qregv \alpha {01} + \qregv \alpha {10} + \qregv \alpha {11} = 
\begin{pmatrix}
    \alpha_{00}  \\
    \alpha_{01}  \\
    \alpha_{10}  \\
    \alpha_{11}  \\
\end{pmatrix}
    \]
    \uncover<2->{
    \[
    |\alpha_{00}|^2 + |\alpha_{01}|^2 + |\alpha_{10}|^2 + |\alpha_{11}|^2 = 1
        \]}
    \end{example}
\end{frame}

\subsection{Quantenschaltkreise}
\begin{frame}{Quantenschaltkreise}
    \begin{itemize}
        \item klassische Gatter können als Wahrheitstabellen dargestellt werden
        \item Quantengatter müssen alle Zustände behandeln
        \item viele weitere Einschränkungen 
        \item nur azyklische Schaltkreise betrachtet
        \item Beispiel: NOT-Gatter
    \end{itemize}
\end{frame}

\begin{frame}[t]{NOT-Gatter}
    \begin{itemize}
        \item \(
            \NOT(\qreg 0) = \qreg 1\) \\
            \(\NOT(\qreg 1) = \qreg 0
            \)

        \pause
        \item Generalisierung: \(
            \NOT(\alpha \qreg 0 + \beta \qreg 1) = \beta \qreg 0 + \alpha \qreg 1
        \)
        \pause
        \item Ausgabe muss wieder Qubit sein
        \item Länge bleibt erhalten: \(
            |\alpha|^2 + |\beta|^2 = 1 
            \)

        \pause
        \item Darstellung als Matrix-Vektor-Multiplikation: \(
            \begin{bmatrix}
                0 & 1 \\
                1 & 0 \\
            \end{bmatrix} \cdot 
            \begin{pmatrix}
                \alpha \\ \beta
            \end{pmatrix} = 
            \begin{pmatrix}
                \beta \\ \alpha
            \end{pmatrix} 
            \)
    \end{itemize}
\end{frame}

\begin{frame}{Quantenschaltkreise - weitere Eigenschaften}
    \begin{itemize}
        \item alle Quantengatter sind als Matrizen darstellbar
        \item Matrizen sind unitär: $U^\dagger U = U U^\dagger = I$ \\ 
            ($U^\dagger$ ist konjugiert transponierte Matrix)
        \begin{itemize}
            \item Matrizen sind quadratisch: Gatter haben gleiche Eingabe- und Ausgabegrö\ss e
            \item Alle Berechnungen sind linear
            \item Alle Schaltkreise sind invertierbar: \\
                Viele Funktionen (Bits kopieren, AND, OR, XOR) nicht direkt umsetzbar
        \end{itemize}
    \end{itemize}
\end{frame}

\begin{frame}[t]{Beispiel: CNOT} 
    Umsetzung durch Kontrollbits
    \begin{example}
        \begin{itemize}
            \item CNOT ist verallgemeinertes XOR

            \item \CNOTFormula

            \only<2>{
                \centerline{ \pictureCNOTa }
            }
            \only<3>{
            \item Als Matrix: 
                \(
                \CNOTMatrix
                \)

            \item CNOT ist unitär, da $\CNOT^\dagger = \CNOT$
            }
        \end{itemize}
    \end{example}
\end{frame}

\begin{frame}{No cloning theorem}
    \begin{block}{Axiom}
        Quantenbits können im Allgemeinen nicht geklont werden.
    \end{block}
    \begin{itemize}
            \pause
        \item kein Gatter $\qregfun{\psi,\varphi}{\psi,\psi}$
        \item nicht durch unitäre Matrix ausdrückbar
        \pause
        \item \alert{scheinbarer Widerspruch:} \\ CNOT mit $\varphi=\qreg 0$ ergibt \qregfun{\psi,0}{\psi, \psi}

    \end{itemize}
\end{frame}

\begin{frame}{No cloning theorem}
    \newcommand{\psiReg}{\alpha \qreg 0 + \beta \qreg 1}
    \begin{itemize}
        \item Sei \( \psi = \alpha \qreg 0 + \beta \qreg 1 \)
        \item Dann gilt: \( \qreg{\psi,0} = \alpha \qreg{00} + \beta \qreg {01} \)
        \item \( \CNOT(\qreg{\psi,0}) = \alpha \qreg{00} + \beta \qreg{11}  \)

        \pause
        \item \alert{Aber:}
            \[
                \begin{split}
                    \qreg{\psi,\psi} &= \left[\psiReg \right]\left[\psiReg \right] \\
                        &= \alpha^2 \qreg{00} + \alpha\beta\qreg{01} + \alpha\beta\qreg{10} + \beta^2\qreg{11}
                \end{split}
            \]
        \pause
        \item Gleichheit gilt nur bei $\alpha=0$ oder $\beta=0$
    \end{itemize}
\end{frame}

\section{Simulation klassischer Computer}

\begin{frame}{Simulation klassischer Computer}
    \begin{theorem}
        Jeder klassische Schaltkreis kann auf einem Quantencomputer effizient simuliert werden.
    \end{theorem}

    Zentrale Rolle dabei spielt das Toffoli-Gatter
\end{frame}

\begin{frame}[t]{Toffoli-Gatter}
    \(
        \qregfun{a, b, c}{a, b, c\oplus (a\wedge b)}
    \)

    \only<1>{ \pictureToffoli }
    \only<2-> {
    \begin{itemize}
        \item invertierbar: zweimal Anwenden ergibt 
            \[
            \qreg{a, b, c\oplus (a\wedge b) \oplus (a\wedge b)} = \qreg{a, b, c}
            \]
        \item Kopieren von Bits: \\ Für $a = \qreg 1, c = \qreg 0$: \qregfun{1,b,0}{1,b,b}
        \item NAND: \\ Für $c = \qreg 1$: \qregfun{a,b,1}{a,b,\neg(a \wedge b)}
        \item Alle anderen Gatter können über NANDs realisiert werden.
    \end{itemize}
    }
\end{frame}

\begin{frame}{Simulation klassischer Computer}
    \begin{itemize}
        \item Mehrere Kabel können an Ausgang angebracht werden
        \item Über verschaltete NANDs ist jede klassische Schaltung realisierbar
        \item Simulation ist effizient: Jedes Gatter wird durch konstant viele Toffoli-Gatter ersetzt
    \end{itemize}
\end{frame}

\begin{frame}{Deutschs Algorithmus}
    \begin{itemize}
        \item einfacher Algorithmus
        \item zeigt Quantenparallelismus
        \item Aber: keine reale Anwendung
    \end{itemize}
\end{frame}

\begin{frame}{Vorbereitung: Hadamard-Gatter}
    \[
        H = \frac{1}{\sqrt 2}
        \begin{bmatrix}
            1 & 1 \\
            1 & -1 \\
        \end{bmatrix}
    \]

    \centerline{ \pictureHadamard }

    \[
    \hadamardPlus
    \]
    \[
    \hadamardMinus
    \]

\end{frame}

\section{Deutschs Algorithmus}
\begin{frame}{Deutschs Algorithmus}
    \begin{itemize}
        \item gegeben: Funktion $f(x) : \{0,1\} \rightarrow \{0,1\}$
        \item entscheide, ob $f(0) = f(1)$
        \item alternativ: Berechne $f(0) \oplus f(1)$
        \item klassischer Algorithmus: Berechne jeweils $f(0)$ und $f(1)$.
        \item Deutschs Algorithmus löst das Problem mit einer Auswertung von $f$.
    \end{itemize}
\end{frame}


\begin{frame}{Deutschs Algorithmus}
    \begin{itemize}
        \item gegeben: Funktion $f(x) : \{0,1\} \rightarrow \{0,1\}$
        \item Sei $U_f$ Quantengatter und setze $\qregfun{x,y}{x, y \oplus f(x)}$ um. 
        \item Für $y = \qreg 0$ kann $f(x)$ berechnet werden.
        \item Stattdessen: $x = \qreg + = \qregPlus$.
        \item Dann $U_f$ anwenden: \[
                \qreg{x,f(x)} = \frac{\qreg{0, f(0)} + \qreg{1, f(1)}}{\sqrt 2}
                \]
        \item \alert{Problem:} Es werden zwar $f(0)$ und $f(1)$ berechnet, aber man erhält beim Messen nur jeweils eines.
    \end{itemize}
\end{frame}

\begin{frame}{Deutschs Algorithmus: Vorüberlegungen}
    \begin{itemize}
        \item Quanteninterferenz
        \item Sei nun wieder $x$ beliebig. \\[0.2em]
            Setze $y = \qreg - = \qregMinus$ und wende $U_f$ an.
        \item 
        \begin{equation*}
            U_f \cdot \left( \qreg x \qbitH - \right) = (-1)^{f(x)} \qreg x \qbitH - 
        \end{equation*}
    \end{itemize}
\end{frame}

\begin{frame}[t]{Deutschs Algorithmus}
    \pictureDeutschAlgo

    \only<1>{
        \begin{equation*}
        \qreg {\psi_1} = \qbitH+\qbitH- 
        \end{equation*}
    }
    \only<2>{
    Anwendung von $U_f$ ergibt 
        \begin{equation*}
            \qreg{\psi_2} = \begin{cases}
                \pm \qbitH+ \qbitH- & \text{für } f(0) = f(1) \\[1.5em]
                \pm \qbitH- \qbitH- & \text{für } f(0) \neq f(1)
                \end{cases}
        \end{equation*}
    }
    \only<3>{
        \begin{equation*}
            \begin{split}
                \qreg{\psi_3} &= \left\{\begin{array}{lr}
                    \pm \qreg 0 \qbitH- & \text{für } f(0) = f(1) \\[1.5em]
                    \pm \qreg 1 \qbitH- & \text{für } f(0) \neq f(1) 
                \end{array}\right\} \\[1.2em] &= \pm \qreg{f(0) \oplus f(1)} \qbitH-
            \end{split}
        \end{equation*}
    }
\end{frame}

\subsection{Deutsch-Jozsa Algorithmus}
\begin{frame}{Deutsch-Jozsa Algorithmus}
    \begin{itemize}
        \item Generalisierung von Deutschs Algorithmus auf $n$ bits.
        \item gegeben: Funktion $f(x) : \{0,1\}^n \rightarrow \{0,1\}$
        \item entscheide, ob $f$ konstant oder balanciert (Hälfte 0, Hälfte 1)\\
            andere Werte treten nicht auf
        \item auf klassischem Rechner: $\Theta(2^n)$
        \item auf Quantenrechner in Linearzeit mit $\Theta(n)$ Qubits
    \end{itemize}
\end{frame}

\section{Komplexität}
\begin{frame}{Komplexität}
    \begin{itemize}
        \item Simulation von klassischen Schaltkreisen ohne Zeitverlust
        \item Deutsch-Jozsa-Algorithmus exponentiell schneller
        \item Polynomialzeit auf Quantenalgorithmen $\cBQP$: \\
            \emph{bounded error quantum polynomial time}
        \item $ \cP \subseteq \cBPP \subseteq \cBQP \subseteq \cPSPACE $
        \item $\cNP~?~\cBQP$
    \end{itemize}
\end{frame}

\section{Zusammenfassung}
\begin{frame}{Zusammenfassung}
    \begin{itemize}
        \item Quantenrechner sind (vermutlich) mächtiger als klassische
        \item Bei Komplexität ist noch vieles unbekannt
        \item noch keine nutzbaren Quantenrechner
        \item experimentelle Systeme mit wenigen Qubits konnten Quantenalgorithmen nutzen
        \item Möglichkeit zur technischen Realisierung
        \item bisher nur wenige Algorithmen
    \end{itemize}
\end{frame}

\end{document}


